\documentclass{article}

\begin{document}

\section*{2a}
Assume peer $p$ and new peer $q$. $p$ can only update if $q < succ(p + i^{m-1})$. The peer at position
$p - 2^{m-1}$ cannot update since its last finger table entry $<= p$. The peer at
position $p + 2^{m-1}$ cannot update since it is past $q$. Thus there are at most
$2\times2^{m-1}=2^m$ finger tables that could update. However if there are any peers $r$
in the range $(q - 2^{m-1}, q)$, none of the peers in the range $(q - 2^m, q - 2^{m-1})$
will update since $r$ will be the last element in the finger table for all of
them. If there are no peers in the range $(q - 2^{m-1}, q)$ then all of the peers in
the range $(q - 2^m, q - 2^{m-1})$ will be forced to update.  Since this if half of
all of the possible updates, and the alternative is to remove half of them, this
must be the worst case scenario of number of updates. So the worst case is $2^{m-1}$.


\section*{2b}
A peer $p$ will only need to be updated if $succ(p + 1)$ has changed due to a new peer $q$ being inserted.
So only if $p < q < succ(p + 1)$.
Take another peer $r$.
Let's say that $p - r > 1$, so $r + 1 < p$.
Therefore $succ(r + 1)$ could be $p$ if there are no other nodes in the way.
Thus the only way inserting $q$ could cause an update to $r$ is if $q < p$.
But the only way it could cause an update to $p$ is if $q > p$.
Therefore $q$ could cause at most 1 lookup table update if $p - r > 1$ for any $p, r$ in peers.
So having 3 lookup table updates is not possible at all, since given 3 nodes two of them must be at least 1 space away.
Having 2 lookup table updates is possible however:\\
\\
$r = 1, succ(r + 1) = 7$\\
$p = 2, succ(p + 1) = 7$\\
\\
node 7\\
node 8\\
node 9\\

Inserting $q$ at key 3 causes $r$ and $p$ to update.
Therefore the worst case scenario is 2 lookup tables updating




\end{document}
